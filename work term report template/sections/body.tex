\section{Introduction}
The introduction introduces the report to the reader by:
\begin{itemize}
	\item making a few background statements about the company/organization
	\item introducing the subject to be discussed
	\item mentioning why the subject is important
	\item outlining the content of the rest of the report.
	\item containing sufficient background information for the reader to understand the rest of the
report.
\end{itemize}
Introductions should never be longer than the discussion. If a significant amount of background information is required, some of the material may be included as appendices.
The introductory material may be presented in several sections to cover the scope of the report as well as provide the necessary background information. In the sample Table of Contents, the introductory portion is contained in sections 1 through 4.

\section{Discussion}
The discussion is the foundation of a report. It presents evidence in the form of referenced facts, data, test results, and analysis upon which the conclusions are based. A well-written discussion flows logically from concept to concept to lead the reader to the appropriate conclusions.
The discussion may contain several sections if several concepts are presented. In the sample Table of Contents, the discussion is contained in subsections 5.1 through 5.5.

\section{Conclusion}
Conclusions are the results derived from the evidence provided in the discussion. No new material is presented in the conclusion.
\subsection{sub conclusion}
When presenting more than one conclusion, state the main conclusion first followed by the others in the order of decreasing importance, to ensure the maximum impact on the reader.

\section{Recommendation}
Recommendations are an outline of what further work needs to be done based solidly on the information you previously presented in the report\cite{einstein}. They have the greatest impact when written using action verbs. Again, do not introduce new material or concepts here\cite{knuth-fa, knuthwebsite}
